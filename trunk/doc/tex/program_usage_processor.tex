\subsection{Processor}
Program \textsc{processor} jest rdzeniem naszego projektu, kt�ry w zamierzeniu mia� w jak naprostszy, oraz jak najszybszy spos�b wygenerowa� 
rozwi�zanie optymalne problemu. Program jest aplikacj� konsolow�, tak wi�c aby skorzysta� z niego samego (bez wykorzystania nak�adki GUI) nale�y
uruchomi� program w konsoli (cmd.exe w windows, b�d� w terminalu pow�oki linuks/unix). Program wywo�ujemy w spos�b nast�puj�cy:
\begin{verbatim}
processor --input=INPUT --silent --K=x --T=y --ALPHA=z
\end{verbatim}
Wszystkie parametry opr�cz pierwszego (input) s� opcjonalne. Poni�ej opis parametr�w programu:
\begin{itemize}
    \item input nazwa pliku wej�ciowego (mapy)
    \item silent - tryb bez dodatkowych komunikat�w oraz informacji - wykorzystywany przez nak�adk� GUI
    \item{K, T, ALPHA} parametry odpowiednio K - ilo�� iteracji, ALPHA - parametr algorytmu, T - ilo�� iteracji przez jak� dany element jest na 
          li�cie tabu
\end{itemize}

