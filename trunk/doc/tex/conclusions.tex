\section{Wnioski}
Z przeprowadzonych test�w wynika, �e program nasz dzia�a poprawnie i proponowane rozwi�zania wygl�daj� poprawnie i zgodnie z oczekiwaniami. Potwierdzi�y si� r�wnie� nasze przypuszczenia o wydajno�ci algorytmu, co pozwala na rozwi�zywanie bardziej skomplikowanych map w rozs�dnym czasie.\\
Metoda Tabu Search okaza�a si� skuteczna i szybka. Prawdopodobnie gdyby�my chcieli wykonywa� obliczenia algorytmem �cis�ym trwa�y by one bardzo d�ugo. Dzi�ki zastosowaniu algorytmu przybli�onego czas ten jest bardzo przyzwoity, dla ma�ych map wr�cz pomijalny (rz�du dziesi�tek milisenkund). Pozwala to s�dzi�, �e zaproponowany przez nas model sprawdza si� i mo�e znale�� bardziej praktyczne zastosowanie.\\
Dodatkowo mo�na nieco zmieni� rozumowanie i rozpatrywa� budynki jako np wioski, co pozwoli na takie rozlokowanie anten, aby przykry� nimi jak najwi�ksz� ilo�� wsi (tak mo�na popatrzy� np. na mapk� "map10").
